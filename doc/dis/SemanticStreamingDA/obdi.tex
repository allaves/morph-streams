

\subsection{Ontology-based data integration}
\label{obdi}
%- focus on integration problems

Data integration can be defined as the process of providing unified and transparent data access to multiple and heterogeneous data sources [Len02]. The problem of integrating heterogeneous data sources involves not only providing access to information but also providing means of interpreting and processing the data in a consistent manner. When dealing with different sources it is often complicated to establish the meaning of the incoming data. And if the meaning cannot be clearly identified, then it is impossible to pair information for the different sources [RAY+00]. Problems of this kind are mainly due to confusion in term meanings and naming conflicts. Added to syntactical conflicts, the semantic integration challenge is hard to tackle.
We can identify the main issues of data integration as [WVV+01]:
\begin{itemize}
	\item Incompatibility in communication and protocols, across different data management systems
	\item Syntactical heterogeneity: differences in data model representation and structure, incompatibility of data values.
	\item Semantic heterogeneity: differences in naming and abstraction level, homonyms and synonyms in schemas.
\end{itemize}.
Ontologies provide a means of explicitly specifying the meaning of the information that will be interchanged between systems. In that way it can help achieve semantic interoperability: regardless of the source's data syntax or semantics, we can map it to a known ontology and access the data in terms of the mediated ontological view [WVV+01]. As a consequence semantic declarative queries can be written in terms of the mediated schema.
It has been acknowledged that ontologies may be useful to solve this semantic heterogeneity problem as it is described in [WVV+01, DH05]. We can mention several systems and prototypes based on ontologies for data integration support: OBSERVER [MIK+00], SIMS [ACH+93], Carnot [HJK+93], DWQ [CDL+98], PICSEL [GLR00], MOMIS [BBV+07].
The main role and purpose of ontologies in an integration system is to explicitly state the semantics of the data sources so that it can be possible to identify and establish semantic relationships between these sources. For instance, if a data source stores person information and another source stores student information, the ontology may represent both concepts, with person subsuming student. This information can later be exploited by query integrators to coherently perform joins or unions over these sources.
Data sources can be relational databases, data files, xml files or any other storing media. Most of the work on data integration has been centred on relational databases, as it is the most widespread way of storing large volumes of information. In any case the ultimate goal is to allow querying the different sources through a uniform interface, hiding the underlying heterogeneous schemas. Therefore the applications querying the integrated system must only focus on the information they wish to obtain, leaving to the integration system the work of searching for the information in the different sources, clearly simplifying the development of upper layer applications on top of the integration infrastructure.
Research on the subject has produced many approaches to building such integration systems. We mention two main alternatives: the \textit{virtualisation} and \textit{materialisation} approaches [IKK05].
In the virtualisation approach queries are posed over the mediated schema and a mediator component identifies the sources that will be needed to produce the answer. Then a series of appropriate sub-queries are automatically written for these sources and executed. The results of each source are retrieved, post-processed and transformed to the mediated schema and returned to the caller application.
This alternative allows answering queries on demand, even if the update rate of data is high and the queries are potentially arbitrary. As the queries are rewritten dynamically for each of the sources, it is possible to retrieve any portion of data exposed through the mediated schema. On the other hand the transformations on the queries and the processing of the data from the sources to the mediated schema may be complex and incur in performance issues. Another potential problem is data source availability, as the queries are performed live; if the source is unreachable then the queries may not be able to be completed. Even if the source is available, latency caused by networking or connection problems may increase the query response time.
The materialisation approach differs from the virtualised one in that it extracts and transforms the data from the sources periodically, instead of performing on-the-fly conversions each time a query is being executed. Instead, relevant data is first identified, and extracted from the multiple data sources in a batch process operation. The extracted data is then stored in a data warehouse where it can be accessed by the external applications. The data warehouse can then be updated periodically depending on the requirements. Under this approach, the time consumed in retrieving the data from the sources is moved to the off-line phase of updating the centralised data warehouse. Therefore when the external applications query the integration system, results can be accessed almost immediately. However the data may be stale in occasions, and it may not be possible to access all possible pieces of data, as the warehouse only stores selected materialised views of the original sources.
In the case of the virtualised approach, a mediator component is introduced as a major feature of the integration architecture. This mediator is in charge of transforming the original query into sub-queries for the sources and then transforming the results back to the mediated schema. This process of transformation is performed thanks to mapping definitions that establish relationships between the mediated and source schemas. 

There are three main alternatives for defining these mappings [Len02, IKK05]: \textit{Global-as-view (GAV)}, \textit{Local-as-view (LAV)} and a combination of both, \textit{GLAV}.
In the LAV approach, each of the source schemas is represented as a view in terms of the global schema. If our system $\mathcal{I}$ is represented as:
$\mathcal{I = <G,S,M>}$ , where $\mathcal{G}$ is the global schema, $\mathcal{S}$ is a source schema and $\mathcal{M}$ is the mapping between $\mathcal{G}$ and $\mathcal{S}$, then the mapping assertions in $\mathcal{M}$ will be a set of elements of the form:
$s \leadsto q_{\mathcal{G}}$
Where $s$ is an element of $\mathcal{S}$ and $q_{\mathcal{G}}$ is a query over the global schema $mathcal{G}$.
This approach is useful if the global schema is well established and if the sources suffer modifications constantly. Notice that changes in the sources do not affect the global schema. However, the query processing is not obvious, as it is not explicitly stated how to obtain the data in the mapping definition. Query rewriting techniques such as query answering using views can be used in this approach [Hal01]. Examples of LAV based works are DWQ [CDL+98], InfoMaster [GKD97], Information Manifold [Lev98], PICSEL [GLR00].
The other approach, GAV, defines the mappings in the opposite way. The global schema elements are represented as views over the source schemas. The mapping definition explicitly defines how to query the sources to obtain the desired information in terms of the global schema. Following the system representation $\mathcal{I = <G,S,M>}$, in the GAV approach the mapping assertions are elements of the form:
$g \leadsto q_{\mathcal{S}}$
Where $g$ is an element of the global schema $\mathcal{G}$ and is expressed as a view $q_{\mathcal{S}}$, a query on the source schema. The advantage is that the mapping itself already indicates how to query the sources to obtain the data, so the processor can directly use this information to perform the query rewriting. The main disadvantage is that in case of changes or additions on the sources -e.g. a new source is added- then the global schema may suffer changes and this may affect other mapping definitions. Examples of GAV based works are SIMS [ACH+93], TSIMMIS [GHI+97], Carnot [HJK+93], Gestalt [RS98], MOMIS [BBV+07], IBIS [CCDG+03], DIS@DIS [CDL+03].
For the sake of completeness we can mention the GLAV (Global-Local-as-view) approach, which is a generalisation of the previous two. This formalism allows more expressive mappings than LAV and GAV combined, and it has been shown to reach the limits of tractability of these description languages [FLM99].
