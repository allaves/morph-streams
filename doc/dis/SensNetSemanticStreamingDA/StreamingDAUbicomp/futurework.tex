\section{Conclusions and Future Work}
\label{conclusions}

We have presented an approach for providing access to streaming data based on ontologies, by extending the \rtwoo\ mapping definition language, the ODEMapster processor and the C-SPARQL language.\ %We have provided details about the underlying technologies that we use as a basis: SPARQL, SNEEql, \rtwoo\ and ODEMapster.
We have presented the \sparqlstr\ extensions to C-SPARQL for RDF streams and the \stwoo\ extensions to \rtwoo\ for
stream mappings. Then we have shown the semantics of the proposed extensions and the mechanism to generate data source
queries from the original ontological queries using the mappings. The case presented here generated SNEEql queries but
the techniques are independent of the target stream query language. Finally the prototype implementation has shown the
feasibility of the approach. This work constitutes a first effort towards ontology-based streaming data integration,
relevant for supporting the increasing number of sensor network applications being developed and deployed in the latest
years. The extensions presented in this paper can be summarised in Table \ref{tab:tabla}.

\begin{table}
%\vspace{-10pt}
\smally
%\scriptsize
\begin{tabular}{|c|c|c|c|}
\hline
Basis & Extension & Syntax & Semantics \\
\hline
\multirow{2}{*}{C-SPARQL} & Window variable %upper boundary 
& \texttt{RANGE} $t_i$ \texttt{TO} $t_f$& $k \centerdot \delta + t_i<\tau $ \\
& upper boundary & & $\leq k \centerdot \delta +t_f$ \\
&&&\\
& Syntax for current & \texttt{NOW} & $\tau = now$ \\
& timestamp &&\\
\hline
\multirow{4}{*}{R\subscript{2}O} & Stream definitions & \texttt{streamschema-desc} & Streaming data types:\\
& in mappings & \texttt{has-stream} & \texttt{StreamOf [Data]} \\
& & \texttt{timestamp-desc} & \texttt{TaggedTuple,}\\
& & & \texttt{Window.} \\
& Virtual RDF  & \texttt{virtualStream} & \texttt{extentName}\\
& Stream IRIs & \texttt{<IRI>} & \\
\hline
\multirow{3}{*}{ODEMapster} & Window translation & - & $\lambda(\Phi_{S_i}(\vec{x})[t_i,t_f,\delta])$\\
&in the processor, &  & $\omega_{t_i,t_f,\delta}S_i$ \\
&classes, object and &  &    \\
& dataype attributes &  &    \\
\hline
\end{tabular}
\vspace{10pt}
\caption{Extensions and additions to \rtwoo , ODEMapster and C-SPARQL}
\label{tab:tabla}
%\vspace{-20pt}
\end{table}



%\section{Future Work}
%\label{future}

Although we have shown initial results querying the underlying SNEE engine with basic queries, we expect to consider in
the near future more complex query expressions including aggregates, combination of time and tuple windows and joins
between streams. We also plan to adapt our query rewriting approach to more recent and promising works such as
\cite{PerezUrbina_09}. We are also aware of the need of optimising the generated queries using techniques from sensor
networks and continuous data approaches \cite{Abadi_2005,Arasu_06a,Galpin_09}. It is also our goal to provide a
characterisation of our algorithms. In the scope of a larger streaming and sensor networks integration framework, we
intend to achieve the following goals: i) integrating streaming and stored data sources through an ontological unified
view; ii) combining data from event-based streams and/or sensor networks acquisitional streams considering time and
triple windows; iii) considering quality-of-service requirements for query optimisation and source selection during the
integration.

The present work can be just considered as a first step to our goal of providing an ontology-based integration platform
for continuous heterogeneous data sources. Therefore we will address the problems of heterogeneity, distributed query
processing and integration as part of this research track.
