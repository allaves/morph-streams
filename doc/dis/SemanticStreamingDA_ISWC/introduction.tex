\section{Introduction}
\label{intro}

Recent advances in wireless communications and sensor technologies have opened the way for deploying networks of interconnected sensing devices capable of ubiquitous data capture, processing and delivery.
Sensor network deployments are expected to increase significantly in the upcoming years because of their advantages and unique features.
Tiny sensors can be installed virtually anywhere and still be reachable thanks to wireless communications. 
Moreover, these devices are inexpensive and can be used for a wide variety of applications including security surveillance, healthcare provision, and environmental monitoring.

As an example, consider a web application which aids an emergency planner to detect and co-ordinate the response to a forest fire in the Castilla y Le\'on region of Spain. 
This involves retrieving relevant data from multiple sources, \eg weather data from \textsc{aemet} (Agencia Espa\~nola de Meteorolog\'ia)\footnote{\url{http://www.aemet.es} accessed 25 June 2010.}, sensor data from sensor networks deployed in the region, and any other relevant sources of data such as the \textsc{esa} satellite imagery providing fire risks\footnote{\url{http://dup.esrin.esa.int/ionia/wfa/index.asp} accessed 25 June 2010.}. 
Typically sources are managed autonomously and model their data according to the needs of the deployment.
To integrate the data requires linking the sources to a common data model so that conditions that are likely to cause a fire can be detected, and presented to the user in terms of their domain, \eg fire risk assessments.
We propose that ontologies can be used as such a common model.
For the scenario presented here, we use an ontology that extends ontologies from \textsc{sweet}\footnote{\url{http://sweet.jpl.nasa.gov/} accessed 25 June 2010.} and the W3C incubator group's semantic sensor network ontology\footnote{\url{http://www.w3.org/2005/Incubator/ssn/wiki/Semantic_Sensor_Network_Ontology} accessed 25 June 2010.}.
%However, such a scenario requires techniques for ontology-based access to streaming data sources.

The work presented in this paper considers advances done by the semantic web and database communities over the last decade. On the one hand, the semantic web research has produced mapping languages and software for enabling ontology-based access to stored data sources, \eg \rtwoo \cite{Barrasa_04} and \dtworq \cite{Bizer_07}.
These systems provide semantic access to traditional (stored) data sources by providing mappings
between the elements in the relational and ontological models \cite{Sahoo_09}.
However, similar solutions for streaming data mapping and querying using ontology-based approaches have not been explored yet.

On the other hand, the database research community have investigated data stream processing where the data is viewed as an append-only sequence of tuples.
Systems such as \stream \cite{Arasu_06a} and Borealis \cite{Abadi_2005} have focused on query evaluation and optimisation over streams with high, variable, data rates.
Other systems such as \snee \cite{Galpin_09} and TinyDB \cite{Madden_05}, have focused on data generated by sensor networks, which tends to be at a lower rate, and query processing in the sensor network where resources are more constrained and energy efficiency is the primary concern.
There have also been proposals for query processing over streaming \rdf data \cite{Bolles_08,Barbieri2010An-Execution-En}.
However there is still no bridging solution that connects these technologies coherently in order to answer the requirements of %
i)~establishing mappings between ontological models and streaming data source schemas, and %
ii)~accessing streaming data sources through queries over ontology models.

In this paper we focus on providing ontology-based access to streaming data sources, including sensor networks, through declarative continuous queries.
We build on the existing work of \rtwoo for enabling ontology-based access to relational data sources, and \snee for query evaluation over streaming and stored data sources.
This constitutes a first step towards a framework for the integration of distributed heterogeneous streaming and stored data sources through ontological models and to the provision of Linked Data for streams~\cite{LePhuoc_09,Page_09,Sequeda_09}.
%\fxnote{AG: Can we drop the last clause of the sentence to do with Linked Data? I'm not convinced it adds to our argument.}
In Section~\ref{sec:background} we provide more detailed descriptions of \rtwoo and stream query processing in order to present the foundations of our approach in Section~\ref{approach}. 
In Section~\ref{syntax} we present the syntactic extensions for \sparql to enable queries over \rdf streams, and present \stwoo for stream-to-ontology mappings. 
The semantics of these extensions are detailed in Section~\ref{semanticsstreaming} and a first implementation of the execution of the streaming data access approach is explained in Section~\ref{execution}.
Related work is discussed in Section \ref{sec:related-work} and our conclusions in Section~\ref{conclusions}. 


%%% Local Variables: 
%%% mode: latex
%%% TeX-master: "rere"
%%% End: 
