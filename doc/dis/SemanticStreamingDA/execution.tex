\section{Implementation and Walkthrough}
\label{execution}

The presented approach of providing ontology-based access to streaming data has been implemented as an extension to the \odemapster processor \cite{Barrasa_04}. 
This implementation generates \sneeql queries that can be executed by the streaming query processor.

\begin{lstlisting}[style=SNEEqlStyle,language=Haskell,float,label=list:schema,caption=Relational schema of the data source.]
windsamples: (sensorid INT PK, ts DATETIME PK, speed FLOAT, direction FLOAT)
sensors: (sensorid INT PK, sensorname CHAR(45), lat FLOAT, long FLOAT)
\end{lstlisting}

Consider the motivating example where a sensor network generates a stream \texttt{windsamples} of wind sensor measurements.
The associated stored information about the sensors, \eg location and type, are stored in a relation \texttt{sensors}.
The schemas are presented in Listing~\ref{list:schema}. Also consider the following ontological view:
\begin{align*}%[style=SNEEqlStyle,language=Haskell,frame=none]
\small
SpeedMeasurement \sqsubseteq\ & Measurement \\
WindSpeedMeasurement \sqsubseteq\ & SpeedMeasurement \\
WindDirectionMeasurement \sqsubseteq\ & Measurement \\
SpeedMeasurement \sqsubseteq\ & \exists hasSpeed \\
Measurement \sqsubseteq\ & \exists isProducedBy.Sensor \\
Sensor \sqsubseteq\ & \exists hasName
\end{align*}
We can define an \stwoo mapping that splits the \texttt{windsamples} stream tuples into instances of two different concepts $WindSpeedMeasurement$ and \textit{WindDirectionMeasurement}. 
Listing~\ref{list:s2o-wind-ex} is an extract of the \stwoo mapping document concerning the $WindSpeedMeasurement$.
The mapping extract defines how to construct the $WindSpeedMeasurement$ ($WindSM$) and $Sensor$ class instances from the \texttt{windsamples} stream and the \texttt{sensors} table: $\Psi_{WindSM}\leadsto \Phi_{\mathtt{windsamples}}$ and $\Psi_{Sensor}\leadsto \Phi_{\mathtt{sensors}}$. 
In the case of the $WindSpeedMeasurement$ the function $f_{WindSM}^{Id}$ produces the \uri's of the instances by concatenating the \texttt{sensorid} and \texttt{ts} attributes.
Now we can pose a query over the ontology using \sparqlstr, for example to obtain the wind speed measurements taken in the last 10 minutes (Listing~\ref{list:query-example}).

\begin{lstlisting}[style=R2OStyle,language=R2O,frame=none,float,label=list:s2o-wind-ex,caption=\stwoo mapping from the data stream \textsf{windsamples} to the ontology concepts $WindSpeedMeasurement$.]
conceptmap-def WindSpeedMeasurement
	virtualStream <http://ssg4env.eu/SensorReadings.srdf>
	uri-as
		concat('ssg4env:WindSM_',windsamples.sensorid,windsamples.ts)
	described-by
		attributemap-def hasSpeed
			operation constant
				has-column windsamples.speed
		dbrelationmap-def isProducedBy
			toConcept Sensor
			joins-via
				condition equals
					has-column sensors.sensorid
					has-column windsamples.sensorid

conceptmap-def Sensor
	uri-as
		concat('ssg4env:Sensor_',sensors.sensorid)
	described-by
		attributemap-def hasSensorid
			operation constant
				has-column sensors.sensorid
\end{lstlisting}

\begin{lstlisting}[style=SPARQLSTRStyle,language=SPARQLSTR,frame=none,float,label=list:query-example,caption=\sparqlstr query which every minute returns the wind speed for the last ten minutes.]
PREFIX fire: <http://www.ssg4env.eu#>
PREFIX rdf: <http://www.w3.org/1999/02/22-rdf-syntax-ns#>
SELECT RSTREAM ?speed
FROM STREAM <www.ssg4env.eu/SensorReadings.srdf> [FROM NOW - 10 MINUTES TO NOW STEP 1 MINUTE]
WHERE {
  ?WindSpeed a fire:WindSpeedMeasurement;
    fire:hasSpeed ?speed;
}
\end{lstlisting}


A class query atom $WindSpeedMeasurement(x)$ and a datatype property atom $hasSpeed(x,z)$ can be extracted from the \sparqlstr query. 
The window specification $[t_s=\textsf{NOW}-10,t_e=\textsf{NOW},\delta=1]$ is also obtained\footnote{For the simplicity of presentation, we assume that the system rewrites all time specifications to minutes.
The implemented system uses milliseconds as the common time unit.}.
The \stwoo mapping defines that $WindSpeedMeasurment$ instances are generated based on the \texttt{sensorid} and \texttt{ts} attributes of the \texttt{windsamples} stream, using a concatenation function to generate each instance \uri.
% Hence the processor will evaluate
%\begin{align*}
%\lambda(\Phi_{\mathtt{windsamples}}(x_{\mathtt{sensorid}},x_{\mathtt{ts}}))[now-10,now,1]) %= \\ \pi_{\mathtt{sensorid,ts}}(\omega_{now-10,now,1}\mathtt{windsamples})
%\end{align*}
Similarly the \stwoo mapping defines that $hasSpeed$ properties are generated from the values of the speed attribute of the \texttt{windsamples} stream. 
The processor will evaluate this as:
\begin{align*}
\lambda(\Phi_{\mathtt{windsamples}}(x_{\mathtt{sensorid}},x_{\mathtt{ts}},z_{\mathtt{speed}})[now-10,now,1]) = \\ \pi_{\mathtt{sensorid,ts,speed}}(\omega_{now-10,now,1} (\mathtt{windsamples}))
\end{align*}
In this case no joins and other selection conditions are needed, and only one stream has to be queried to produce the results. 
The query generated in the \sneeql language is the shown in Listing~\ref{list:query-ex-sneeql}\footnote {Although the current available implementation of the SNEE processor lacks the \texttt{concat} operator, we include the sample query in its complete form here.}.

\begin{lstlisting}[style=SNEEqlStyle,language=SNEEql,frame=none,float,label=list:query-ex-sneeql,caption=The \sneeql query that is generated for the input query in Listing~\ref{list:query-example}.]
SELECT RSTREAM concat('http://ssg4env.eu#WindSM',windsensor.id,windsensor.ts) AS id, windsamples.speed AS speed
FROM windsamples[FROM NOW - 10 MINUTES TO NOW SLIDE 1 MINUTE];
\end{lstlisting}



The relational answer stream that results from evaluating the query in Listing~\ref{list:query-ex-sneeql} are transformed by the \textsf{Data Transformation} module depicted in Figure~\ref{fig:SemanticIntegrator} according to the \stwoo mappings.
This results in a stream of tagged triples which are instances of the class $WindSpeedMeasurement$.

%Show transformation example, mention implementation in infancy.
%We need to code at this point and show minimal results.

%%% Local Variables: 
%%% mode: latex
%%% TeX-master: "rere"
%%% End: 
